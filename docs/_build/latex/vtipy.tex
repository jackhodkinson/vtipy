%% Generated by Sphinx.
\def\sphinxdocclass{report}
\documentclass[letterpaper,10pt,english]{sphinxmanual}
\ifdefined\pdfpxdimen
   \let\sphinxpxdimen\pdfpxdimen\else\newdimen\sphinxpxdimen
\fi \sphinxpxdimen=.75bp\relax

\PassOptionsToPackage{warn}{textcomp}
\usepackage[utf8]{inputenc}
\ifdefined\DeclareUnicodeCharacter
 \ifdefined\DeclareUnicodeCharacterAsOptional
  \DeclareUnicodeCharacter{"00A0}{\nobreakspace}
  \DeclareUnicodeCharacter{"2500}{\sphinxunichar{2500}}
  \DeclareUnicodeCharacter{"2502}{\sphinxunichar{2502}}
  \DeclareUnicodeCharacter{"2514}{\sphinxunichar{2514}}
  \DeclareUnicodeCharacter{"251C}{\sphinxunichar{251C}}
  \DeclareUnicodeCharacter{"2572}{\textbackslash}
 \else
  \DeclareUnicodeCharacter{00A0}{\nobreakspace}
  \DeclareUnicodeCharacter{2500}{\sphinxunichar{2500}}
  \DeclareUnicodeCharacter{2502}{\sphinxunichar{2502}}
  \DeclareUnicodeCharacter{2514}{\sphinxunichar{2514}}
  \DeclareUnicodeCharacter{251C}{\sphinxunichar{251C}}
  \DeclareUnicodeCharacter{2572}{\textbackslash}
 \fi
\fi
\usepackage{cmap}
\usepackage[T1]{fontenc}
\usepackage{amsmath,amssymb,amstext}
\usepackage{babel}
\usepackage{times}
\usepackage[Bjarne]{fncychap}
\usepackage{sphinx}

\usepackage{geometry}

% Include hyperref last.
\usepackage{hyperref}
% Fix anchor placement for figures with captions.
\usepackage{hypcap}% it must be loaded after hyperref.
% Set up styles of URL: it should be placed after hyperref.
\urlstyle{same}

\addto\captionsenglish{\renewcommand{\figurename}{Fig.}}
\addto\captionsenglish{\renewcommand{\tablename}{Table}}
\addto\captionsenglish{\renewcommand{\literalblockname}{Listing}}

\addto\captionsenglish{\renewcommand{\literalblockcontinuedname}{continued from previous page}}
\addto\captionsenglish{\renewcommand{\literalblockcontinuesname}{continues on next page}}

\addto\extrasenglish{\def\pageautorefname{page}}

\setcounter{tocdepth}{1}



\title{vtipy Documentation}
\date{Nov 19, 2018}
\release{0.1}
\author{Jack Hodkinson}
\newcommand{\sphinxlogo}{\vbox{}}
\renewcommand{\releasename}{Release}
\makeindex

\begin{document}

\maketitle
\sphinxtableofcontents
\phantomsection\label{\detokenize{index::doc}}



\chapter{vtipy package}
\label{\detokenize{vtipy_docs/vtipy:vtipy-package}}\label{\detokenize{vtipy_docs/vtipy::doc}}\label{\detokenize{vtipy_docs/vtipy:welcome-to-vtipy-s-documentation}}

\section{vtipy.impedance module}
\label{\detokenize{vtipy_docs/vtipy:vtipy-impedance-module}}\label{\detokenize{vtipy_docs/vtipy:module-vtipy.impedance}}\index{vtipy.impedance (module)}\index{solartron1260 (class in vtipy.impedance)}

\begin{fulllineitems}
\phantomsection\label{\detokenize{vtipy_docs/vtipy:vtipy.impedance.solartron1260}}\pysiglinewithargsret{\sphinxbfcode{\sphinxupquote{class }}\sphinxcode{\sphinxupquote{vtipy.impedance.}}\sphinxbfcode{\sphinxupquote{solartron1260}}}{\emph{Vac=0.5}, \emph{Vdc=0.0}, \emph{integration\_time=1}, \emph{gpib\_address=2}}{}
Bases: \sphinxcode{\sphinxupquote{Gpib.Gpib.Gpib}}
\begin{quote}\begin{description}
\item[{Parameters}] \leavevmode\begin{itemize}
\item {} 
\sphinxstyleliteralstrong{\sphinxupquote{Vac}} (\sphinxstyleliteralemphasis{\sphinxupquote{float}}\sphinxstyleliteralemphasis{\sphinxupquote{, }}\sphinxstyleliteralemphasis{\sphinxupquote{optional}}) \textendash{} A.C. voltage for impedance measurements.

\item {} 
\sphinxstyleliteralstrong{\sphinxupquote{Vdc}} (\sphinxstyleliteralemphasis{\sphinxupquote{float}}\sphinxstyleliteralemphasis{\sphinxupquote{, }}\sphinxstyleliteralemphasis{\sphinxupquote{optional}}) \textendash{} D.C. bias.

\item {} 
\sphinxstyleliteralstrong{\sphinxupquote{integration\_time}} (\sphinxstyleliteralemphasis{\sphinxupquote{int}}\sphinxstyleliteralemphasis{\sphinxupquote{, }}\sphinxstyleliteralemphasis{\sphinxupquote{optional}}) \textendash{} Integration time in seconds.

\item {} 
\sphinxstyleliteralstrong{\sphinxupquote{gpib\_address}} (\sphinxstyleliteralemphasis{\sphinxupquote{int}}\sphinxstyleliteralemphasis{\sphinxupquote{, }}\sphinxstyleliteralemphasis{\sphinxupquote{optional}}\sphinxstyleliteralemphasis{\sphinxupquote{,}}) \textendash{} Gpib address of the instrument. Default is 2.

\end{itemize}

\end{description}\end{quote}
\paragraph{Example}

Here is a demonstration of how the object is initialised
and how a frequency dependent measurement is executed.

\fvset{hllines={, ,}}%
\begin{sphinxVerbatim}[commandchars=\\\{\}]
\PYG{g+gp}{\PYGZgt{}\PYGZgt{}\PYGZgt{} }\PYG{k+kn}{from} \PYG{n+nn}{vtipy} \PYG{k}{import} \PYG{n}{solartron1260}
\PYG{g+gp}{\PYGZgt{}\PYGZgt{}\PYGZgt{} }\PYG{n}{sol} \PYG{o}{=} \PYG{n}{solartron1260}\PYG{p}{(}\PYG{p}{)}
\PYG{g+gp}{\PYGZgt{}\PYGZgt{}\PYGZgt{} }\PYG{n}{sol}\PYG{o}{.}\PYG{n}{measure\PYGZus{}impedance}\PYG{p}{(} \PYG{n}{scan\PYGZus{}number} \PYG{o}{=} \PYG{l+m+mi}{1} \PYG{p}{)}
\end{sphinxVerbatim}
\index{measure\_frequency() (vtipy.impedance.solartron1260 method)}

\begin{fulllineitems}
\phantomsection\label{\detokenize{vtipy_docs/vtipy:vtipy.impedance.solartron1260.measure_frequency}}\pysiglinewithargsret{\sphinxbfcode{\sphinxupquote{measure\_frequency}}}{\emph{frequency}}{}
Sends an instruction to the Solarton to measures impedance
at the specified frequency.
\begin{quote}\begin{description}
\item[{Parameters}] \leavevmode
\sphinxstyleliteralstrong{\sphinxupquote{frequency}} (\sphinxstyleliteralemphasis{\sphinxupquote{float}}) \textendash{} frequency for measurement.

\item[{Returns}] \leavevmode
\sphinxstylestrong{result} \textendash{} Impedance data formatted as a comma separated
string. The first three values (frequency, magnitude, and the
argument) are the important values that need to be saved.

\item[{Return type}] \leavevmode
string

\end{description}\end{quote}

\end{fulllineitems}

\index{measure\_impedance() (vtipy.impedance.solartron1260 method)}

\begin{fulllineitems}
\phantomsection\label{\detokenize{vtipy_docs/vtipy:vtipy.impedance.solartron1260.measure_impedance}}\pysiglinewithargsret{\sphinxbfcode{\sphinxupquote{measure\_impedance}}}{\emph{scan\_number}, \emph{filename=None}, \emph{fmin=0.05}, \emph{fmax=9800000.0}, \emph{ppd=20}, \emph{Tset=None}, \emph{Tcell=None}, \emph{scan\_label=None}}{}
Sends an instructions to the Solartron to make an impedance measurement
over a range of frequencies. This is accomplished by executing the
solartron1260.measure\_impedance\_process method as a subprocess as to
allow a main control script to continue to monitor temperature while
this runs in parallel.
\begin{quote}\begin{description}
\item[{Parameters}] \leavevmode\begin{itemize}
\item {} 
\sphinxstyleliteralstrong{\sphinxupquote{scan\_number}} (\sphinxstyleliteralemphasis{\sphinxupquote{int}}\sphinxstyleliteralemphasis{\sphinxupquote{, }}\sphinxstyleliteralemphasis{\sphinxupquote{optional}}) \textendash{} This should be specified as to indicate the relative order of the
measurement.

\item {} 
\sphinxstyleliteralstrong{\sphinxupquote{filename}} (\sphinxstyleliteralemphasis{\sphinxupquote{None}}\sphinxstyleliteralemphasis{\sphinxupquote{, }}\sphinxstyleliteralemphasis{\sphinxupquote{string}}) \textendash{} The name of file in which impedance data will be
stored. If left as None it will be assigned a name based on the
scan number: “scan\_n.txt”, where n = scan\_number.

\item {} 
\sphinxstyleliteralstrong{\sphinxupquote{fmin}} (\sphinxstyleliteralemphasis{\sphinxupquote{float}}\sphinxstyleliteralemphasis{\sphinxupquote{, }}\sphinxstyleliteralemphasis{\sphinxupquote{optional}}) \textendash{} The minimum frequency of the impedance sweep. The default value
is 0.05.

\item {} 
\sphinxstyleliteralstrong{\sphinxupquote{fmax}} (\sphinxstyleliteralemphasis{\sphinxupquote{float}}\sphinxstyleliteralemphasis{\sphinxupquote{, }}\sphinxstyleliteralemphasis{\sphinxupquote{optional}}) \textendash{} The maximum frequency of the impedance sweep. The default value
is 9.8e6.

\item {} 
\sphinxstyleliteralstrong{\sphinxupquote{ppd}} (\sphinxstyleliteralemphasis{\sphinxupquote{int}}\sphinxstyleliteralemphasis{\sphinxupquote{, }}\sphinxstyleliteralemphasis{\sphinxupquote{optional}}) \textendash{} The number of points per decade at which an impedance measurement
will be recorded. Default is 20.

\item {} 
\sphinxstyleliteralstrong{\sphinxupquote{Tset}} (\sphinxstyleliteralemphasis{\sphinxupquote{None}}\sphinxstyleliteralemphasis{\sphinxupquote{, }}\sphinxstyleliteralemphasis{\sphinxupquote{int}}) \textendash{} The setpoint temperature from the Eurotherm. If set
it is stored in the header of the datafile. It should be set
during variable temperature experiments. If left as none a NAN
value will be recorded in the header.

\item {} 
\sphinxstyleliteralstrong{\sphinxupquote{Tcell}} (\sphinxstyleliteralemphasis{\sphinxupquote{None}}\sphinxstyleliteralemphasis{\sphinxupquote{, }}\sphinxstyleliteralemphasis{\sphinxupquote{int}}) \textendash{} The “cell temperature” as measured by the MAX31855 thermocouple.
If set it is stored in the header of the datafile. If left as none
a NAN value will be recorded in the header.

\item {} 
\sphinxstyleliteralstrong{\sphinxupquote{scan\_label}} (\sphinxstyleliteralemphasis{\sphinxupquote{None}}\sphinxstyleliteralemphasis{\sphinxupquote{, }}\sphinxstyleliteralemphasis{\sphinxupquote{string}}) \textendash{} Optional label to provide extra meta-data for later analysis.
Typically this is set to “up” or “down” to distinguish measurements
during an upward or downward temperature sweep. This value is
recorded in the header of the datafile.

\end{itemize}

\end{description}\end{quote}

\end{fulllineitems}

\index{measure\_impedance\_process() (vtipy.impedance.solartron1260 method)}

\begin{fulllineitems}
\phantomsection\label{\detokenize{vtipy_docs/vtipy:vtipy.impedance.solartron1260.measure_impedance_process}}\pysiglinewithargsret{\sphinxbfcode{\sphinxupquote{measure\_impedance\_process}}}{\emph{scan\_number}, \emph{filename}, \emph{fmin}, \emph{fmax}, \emph{ppd}, \emph{Tset}, \emph{Tcell}, \emph{scan\_label}}{}
Instructs the Solartron to measure the impedance over a range of
frequencies.

This method is designed to be run in parallel with a main script as to
allow the main script to continue to monitor process temperatures
while the impedance is measured. This is accomplished by using the
solartron1260.measure\_impedance method which calls this method (the
solartron1260.measure\_impedance\_process) as a sub-process. However,
this method may be run independently if no other measurements are
required to be run in parallel.

The parameters of the method are the same as those detailed in the
solartron1260.measure\_impedance method, however, the default values
set for that method are not set here. They all must be provided
explicitly.
\begin{quote}\begin{description}
\item[{Parameters}] \leavevmode\begin{itemize}
\item {} 
\sphinxstyleliteralstrong{\sphinxupquote{scan\_number}} (\sphinxstyleliteralemphasis{\sphinxupquote{int}}) \textendash{} This should be specified as to indicate the relative order of the
measurement.

\item {} 
\sphinxstyleliteralstrong{\sphinxupquote{filename}} (\sphinxstyleliteralemphasis{\sphinxupquote{None}}\sphinxstyleliteralemphasis{\sphinxupquote{, }}\sphinxstyleliteralemphasis{\sphinxupquote{string}}) \textendash{} The name of file in which impedance data will be
stored. If set as None it will be assigned a name based on the
scan number: “scan\_n.txt”, where n = scan\_number.

\item {} 
\sphinxstyleliteralstrong{\sphinxupquote{fmin}} (\sphinxstyleliteralemphasis{\sphinxupquote{float}}) \textendash{} The minimum frequency (in Hz) of the impedance sweep.
Cation should be taken when setting this value to anything less
than 0.05 Hz as errors with the Gpib.Gpib library are encountered.

\item {} 
\sphinxstyleliteralstrong{\sphinxupquote{fmax}} (\sphinxstyleliteralemphasis{\sphinxupquote{float}}) \textendash{} The maximum frequency (in Hz) of the impedance sweep. Max value is
1e7.

\item {} 
\sphinxstyleliteralstrong{\sphinxupquote{ppd}} (\sphinxstyleliteralemphasis{\sphinxupquote{int}}) \textendash{} The number of points per decade at which an impedance measurement
will be recorded.

\item {} 
\sphinxstyleliteralstrong{\sphinxupquote{Tset}} (\sphinxstyleliteralemphasis{\sphinxupquote{int}}) \textendash{} The setpoint temperature from the Eurotherm. The value is stored
in the header of the datafile.

\item {} 
\sphinxstyleliteralstrong{\sphinxupquote{Tcell}} (\sphinxstyleliteralemphasis{\sphinxupquote{int}}) \textendash{} The “cell temperature” as measured by the MAX31855 thermocouple.
The value is stored in the header of the datafile.

\item {} 
\sphinxstyleliteralstrong{\sphinxupquote{scan\_label}} (\sphinxstyleliteralemphasis{\sphinxupquote{string}}) \textendash{} A label to provide extra meta-data for later analysis.
Typically this is set to “up” or “down” to distinguish measurements
during an upward or downward temperature sweep. This value is
recorded in the header of the datafile.

\end{itemize}

\end{description}\end{quote}

\end{fulllineitems}

\index{send() (vtipy.impedance.solartron1260 method)}

\begin{fulllineitems}
\phantomsection\label{\detokenize{vtipy_docs/vtipy:vtipy.impedance.solartron1260.send}}\pysiglinewithargsret{\sphinxbfcode{\sphinxupquote{send}}}{\emph{msg}}{}
Sends a message to the solartron. Just like the Gpib.Gpib.write except
a 0.02 s sleep time is applied to prevent too many signals being sent
at the same time.
\begin{quote}\begin{description}
\item[{Parameters}] \leavevmode
\sphinxstyleliteralstrong{\sphinxupquote{msg}} (\sphinxstyleliteralemphasis{\sphinxupquote{string}}) \textendash{} The message to send the solartron. This should be one
of the instructions from the instruction set found in the
solartron manual.

\item[{Returns}] \leavevmode


\item[{Return type}] \leavevmode
None

\end{description}\end{quote}

\end{fulllineitems}


\end{fulllineitems}



\section{vtipy.temperature module}
\label{\detokenize{vtipy_docs/vtipy:vtipy-temperature-module}}\label{\detokenize{vtipy_docs/vtipy:module-vtipy.temperature}}\index{vtipy.temperature (module)}
This module is designed to control the temperature controllers used in the
variable temperature impedance control system.
\index{temperature\_controllers (class in vtipy.temperature)}

\begin{fulllineitems}
\phantomsection\label{\detokenize{vtipy_docs/vtipy:vtipy.temperature.temperature_controllers}}\pysiglinewithargsret{\sphinxbfcode{\sphinxupquote{class }}\sphinxcode{\sphinxupquote{vtipy.temperature.}}\sphinxbfcode{\sphinxupquote{temperature\_controllers}}}{\emph{filename}, \emph{serial\_usb\_port}, \emph{addr=1}, \emph{CLK=25}, \emph{CS=24}, \emph{DO=18}}{}
Bases: \sphinxcode{\sphinxupquote{minimalmodbus.Instrument}}, \sphinxcode{\sphinxupquote{Adafruit\_MAX31855.MAX31855.MAX31855}}

This class controls both the MAX31855 temperature controller
and the Eurotherm 3216.

Methods are defined to allow measurement and recording of the temperature
from the Eurotherm and the MAX31855 as well as to change the setpoint
tempreature of the Eurotherm controller.

This control is accomplished by inheritance from the
minimalmodbus.Instrument and MAX31855.MAX31855 classes.
\begin{quote}\begin{description}
\item[{Parameters}] \leavevmode\begin{itemize}
\item {} 
\sphinxstyleliteralstrong{\sphinxupquote{filename}} (\sphinxstyleliteralemphasis{\sphinxupquote{string}}) \textendash{} The name assigned to the file in which temperature measurements will be
recorded.

\item {} 
\sphinxstyleliteralstrong{\sphinxupquote{serial\_usb\_port}} (\sphinxstyleliteralemphasis{\sphinxupquote{string}}) \textendash{} The path associated with the serial-to-USB converter on the Raspberry Pi
filesystem.

\item {} 
\sphinxstyleliteralstrong{\sphinxupquote{addr}} (\sphinxstyleliteralemphasis{\sphinxupquote{int}}\sphinxstyleliteralemphasis{\sphinxupquote{, }}\sphinxstyleliteralemphasis{\sphinxupquote{optional}}) \textendash{} The slave address of the serial-to-USB converter (connected to the
Eurotherm controller).

\item {} 
\sphinxstyleliteralstrong{\sphinxupquote{CLK}} (\sphinxstyleliteralemphasis{\sphinxupquote{int}}\sphinxstyleliteralemphasis{\sphinxupquote{, }}\sphinxstyleliteralemphasis{\sphinxupquote{optional}}) \textendash{} I/O port on the Raspberry Pi associated with the clock pin on the
MAX31855.

\item {} 
\sphinxstyleliteralstrong{\sphinxupquote{CS}} (\sphinxstyleliteralemphasis{\sphinxupquote{int}}\sphinxstyleliteralemphasis{\sphinxupquote{, }}\sphinxstyleliteralemphasis{\sphinxupquote{optional}}) \textendash{} I/O port on the Raspberry Pi associated with the “chip sellect” port
on the MAX31855.

\item {} 
\sphinxstyleliteralstrong{\sphinxupquote{DO}} (\sphinxstyleliteralemphasis{\sphinxupquote{int}}\sphinxstyleliteralemphasis{\sphinxupquote{, }}\sphinxstyleliteralemphasis{\sphinxupquote{optional}}) \textendash{} I/O port on the Raspberry Pi associated with the “data out” port on the
MAX31855.

\end{itemize}

\end{description}\end{quote}
\paragraph{Example}

The following is an example of how to initialise the temeprature controller
object, make an initial temperature measurement, ramp the temperature,
and hold the temperature at a desired setpoint.

\fvset{hllines={, ,}}%
\begin{sphinxVerbatim}[commandchars=\\\{\}]
\PYG{g+gp}{\PYGZgt{}\PYGZgt{}\PYGZgt{} }\PYG{k+kn}{from} \PYG{n+nn}{vtipy} \PYG{k}{import} \PYG{n}{temperature\PYGZus{}controllers}
\PYG{g+gp}{\PYGZgt{}\PYGZgt{}\PYGZgt{} }\PYG{n}{tc} \PYG{o}{=} \PYG{n}{temperature\PYGZus{}controllers}\PYG{p}{(}\PYG{l+s+s2}{\PYGZdq{}}\PYG{l+s+s2}{temp.txt}\PYG{l+s+s2}{\PYGZdq{}}\PYG{p}{,} \PYG{n}{serial\PYGZus{}usb\PYGZus{}port}\PYG{p}{)}
\PYG{g+gp}{\PYGZgt{}\PYGZgt{}\PYGZgt{} }\PYG{n}{Tcell} \PYG{o}{=} \PYG{n}{tc}\PYG{o}{.}\PYG{n}{measure\PYGZus{}temperatures}\PYG{p}{(}\PYG{p}{)}
\PYG{g+gp}{\PYGZgt{}\PYGZgt{}\PYGZgt{} }\PYG{n}{tc}\PYG{o}{.}\PYG{n}{ramp\PYGZus{}temperature}\PYG{p}{(} \PYG{n}{Tset} \PYG{o}{=} \PYG{l+m+mi}{100} \PYG{p}{)}
\PYG{g+gp}{\PYGZgt{}\PYGZgt{}\PYGZgt{} }\PYG{n}{tc}\PYG{o}{.}\PYG{n}{hold}\PYG{p}{(} \PYG{n}{hold\PYGZus{}time} \PYG{o}{=} \PYG{l+m+mi}{60}\PYG{p}{)}
\end{sphinxVerbatim}
\index{hold() (vtipy.temperature.temperature\_controllers method)}

\begin{fulllineitems}
\phantomsection\label{\detokenize{vtipy_docs/vtipy:vtipy.temperature.temperature_controllers.hold}}\pysiglinewithargsret{\sphinxbfcode{\sphinxupquote{hold}}}{\emph{hold\_time}, \emph{temp\_resolution=2}}{}
This function holds the setpoint temperature of the Eurotherm
while measuring and recording the process temperatures at the
specified temperature resolution.
\begin{quote}\begin{description}
\item[{Parameters}] \leavevmode\begin{itemize}
\item {} 
\sphinxstyleliteralstrong{\sphinxupquote{hold\_time}} (\sphinxstyleliteralemphasis{\sphinxupquote{int}}) \textendash{} Time in minutes to hold the Eurotherm setpoint.

\item {} 
\sphinxstyleliteralstrong{\sphinxupquote{temp\_resolution}} (\sphinxstyleliteralemphasis{\sphinxupquote{int}}) \textendash{} Temperature measurements per minute.

\end{itemize}

\end{description}\end{quote}

\end{fulllineitems}

\index{measure\_temperatures() (vtipy.temperature.temperature\_controllers method)}

\begin{fulllineitems}
\phantomsection\label{\detokenize{vtipy_docs/vtipy:vtipy.temperature.temperature_controllers.measure_temperatures}}\pysiglinewithargsret{\sphinxbfcode{\sphinxupquote{measure\_temperatures}}}{}{}
This method serves two purposes: Firstly, it measures the two cell and
furnace temperature. Secondly, this method records time since the
initial temperature measurement (dt), the setpoint temperature from
the Eurotherm.

The measurement is recorded into the temperature datafile specified
when an instance of this class is defined. The data is added to the
file as a comma separated line of the format:
\begin{quote}

” dt, Tsetpoint, Teuro, Tcell, Ttc\_internal “
\end{quote}
\begin{quote}\begin{description}
\item[{Returns}] \leavevmode
\sphinxstylestrong{Tcell}

\item[{Return type}] \leavevmode
string

\end{description}\end{quote}

\end{fulllineitems}

\index{ramp\_temperature() (vtipy.temperature.temperature\_controllers method)}

\begin{fulllineitems}
\phantomsection\label{\detokenize{vtipy_docs/vtipy:vtipy.temperature.temperature_controllers.ramp_temperature}}\pysiglinewithargsret{\sphinxbfcode{\sphinxupquote{ramp\_temperature}}}{\emph{Tset}, \emph{ramp\_time=1}}{}~\begin{quote}\begin{description}
\item[{Parameters}] \leavevmode\begin{itemize}
\item {} 
\sphinxstyleliteralstrong{\sphinxupquote{Tset}} (\sphinxstyleliteralemphasis{\sphinxupquote{int}}) \textendash{} Setpoint temperature in degrees C.

\item {} 
\sphinxstyleliteralstrong{\sphinxupquote{ramp\_time}} (\sphinxstyleliteralemphasis{\sphinxupquote{int}}) \textendash{} Time (in min) per degree C.

\end{itemize}

\end{description}\end{quote}

\end{fulllineitems}

\index{read\_process\_temperature() (vtipy.temperature.temperature\_controllers method)}

\begin{fulllineitems}
\phantomsection\label{\detokenize{vtipy_docs/vtipy:vtipy.temperature.temperature_controllers.read_process_temperature}}\pysiglinewithargsret{\sphinxbfcode{\sphinxupquote{read\_process\_temperature}}}{}{}
Returns the  furnace temperature from the Eurotherm

\end{fulllineitems}

\index{read\_setpoint() (vtipy.temperature.temperature\_controllers method)}

\begin{fulllineitems}
\phantomsection\label{\detokenize{vtipy_docs/vtipy:vtipy.temperature.temperature_controllers.read_setpoint}}\pysiglinewithargsret{\sphinxbfcode{\sphinxupquote{read\_setpoint}}}{}{}
Reads the setpoint temperature from the Eurotherm. This is read from
register 5 of the Eurotherm 3216. This might need to be adapted for
other Eurotherm controllers. See the Eurotherm manual.
\begin{quote}\begin{description}
\item[{Returns}] \leavevmode
The setpoint temperature from the Eurotherm as a string.

\item[{Return type}] \leavevmode
setpoint (string?)

\end{description}\end{quote}

\end{fulllineitems}

\index{set\_setpoint() (vtipy.temperature.temperature\_controllers method)}

\begin{fulllineitems}
\phantomsection\label{\detokenize{vtipy_docs/vtipy:vtipy.temperature.temperature_controllers.set_setpoint}}\pysiglinewithargsret{\sphinxbfcode{\sphinxupquote{set\_setpoint}}}{\emph{Tset}}{}
Sets the setpoint temperature of the Eurotherm. This is accomplished by
setting register 24 of the Eurotherm 3216. This might need to be
adapted for other Eurotherm controllers. See the Eurotherm manual.
\begin{quote}\begin{description}
\item[{Parameters}] \leavevmode
\sphinxstyleliteralstrong{\sphinxupquote{Tset}} (\sphinxstyleliteralemphasis{\sphinxupquote{int}}) \textendash{} The new setpoint temperature (in degrees Celsius)
for the Eurotherm controller.

\end{description}\end{quote}

\end{fulllineitems}


\end{fulllineitems}



\renewcommand{\indexname}{Python Module Index}
\begin{sphinxtheindex}
\def\bigletter#1{{\Large\sffamily#1}\nopagebreak\vspace{1mm}}
\bigletter{v}
\item {\sphinxstyleindexentry{vtipy.impedance}}\sphinxstyleindexpageref{vtipy_docs/vtipy:\detokenize{module-vtipy.impedance}}
\item {\sphinxstyleindexentry{vtipy.temperature}}\sphinxstyleindexpageref{vtipy_docs/vtipy:\detokenize{module-vtipy.temperature}}
\end{sphinxtheindex}

\renewcommand{\indexname}{Index}
\printindex
\end{document}